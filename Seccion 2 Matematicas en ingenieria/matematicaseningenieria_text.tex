\section{Introducción}
En este número vamos a hacer un estudio matemático en un nuevo tema: la epidemiología. En esta disciplina, nuevamente, la matemática juega un papel muy importante tanto para el estudio de muchas enfermedades como para  elegir estrategias y buenas conductas para poner freno o minimizar consecuencias de algunas epidemias. El primer matemático que dio su primera aportación matemática sobre epidemiología fue Bernoulli (1700-1782) en un artículo que daba un modelo matemático para la viruela de como la vacunación a base de inoculacion de pus en el organismo era eficaz (véase Ref1, ref 2). Pero no ha sido  hasta el siglo XX que se ha empezado a desarrollar modelos epidemiológicos determinísticos o basados en patrones. Podemos destacar a William Heaton(1862-1936) que formuló un modelo discreto para la epidemia de sarampión, a Anderson Gray McKendrick(1876-1943) y William Ogilvy Kermack(1898-1970) quienes formularon modelos epidemilógicos matemáticos deterministas usados para grandes poblaciones y representados a través de ecuaciones diferenciales. Y por último destacar a Ronald Ross(1857-1932) quien formuló que la transmisión de la malaria senreduciría haciendo desaparecer la población de mosquitos y quien ganó el premio Nobel de Fisiología y Medicina por tal trabajo.
\begin{figure}
\begin{figurebox}\centering
\includegraphics[scale=0.35]{Bernoulli.jpg}
%\includegraphics[scale=0.4]{hamer .jpg}  encontrar a este hombre a la web
 \includegraphics[scale=0.1]{william.jpg}
 \includegraphics[scale=1.18]{Anderson.jpg} 
\includegraphics[scale=0.49]{Ronald_Ross.jpg} \caption{De izquierda a derecha: Bernoulli, A.G. McKendrick, W.O. kermack y R. Ross}
\end{figurebox}
%\caption{De izquierda a derecha: Bernoulli, A.G. McKendrick, W.O. kermack y R. Ross}
\end{figure}
El estudio de cada epidemiologia va a venir afectado por caracterícticas intrínsecas de la epidemia. En general, no se puede realizar el mismo estudio para una epidemia originada por una bacteria que por un virus. En el caso de un virús, en general, los individuaos que se han recuperado de la end fermedad han creado una resistencia ante la enfermedad en el que no pueden volver a ser infectados. Pero, en cambio, en el caso de una bacteria los individuos que se repcuperan de la enfermedad puden ser succetibles a volver a enfermarse. Este tipo de diferencias hace que los modelos que podamos construir matemáticamente difieren unos de otros. 

Vamos a distinguir los siguientes tipos de estados que un individuo puede encontrarse intentado cubrir todas las enfermedades posibles:
\begin{itemize}
\item Proporción de individuos sanos y succetibles de ser infectados que denotamos por \textbf{S}.
\item Proporción de individuos infectados que no pueden contagiar a los demás (estado latente) y que denotamos por \textbf{E}.
\item Proporción de individuos infectados y que pueden infectar a otros, denotamos pot \textbf{I}.
\item Proporción de individuos resistentes a la enfermedad, denominamos por \textbf{R}.
\item Proporción de individuos con inmunidad temporal y que denotamos por \textbf{M}.
\end{itemize}

De esta manera dependiendo de los agentes infecciosos por los que se transmite la enfermedad los individuos podrán estar en un estado o en otro. El modo de transmisión de la enfermedad va as er otro factoer importante para t6ener en cuenta en el modelo que se plantee. Algunos ejemplos de transmisión son de persona a persona (como es la gripe), por insectos como es la malaria o por el medio ambiente como lo es el cólera.

Pero un punto importante cuando se estudia una epidemia es si esta perdurará un tiempo en la población o si desaparecerá paulatinamente. Esto es realizar un estudio si la epidemia será endémica o no.
Los modelos matemáticos usados para epidemiología suelen distinguirse entre aquellos determinísticos que consideran a los individuos pertenecientes a un estado del modelo y modelos centrados en el individuo que estudian el comportamiento de cada individuo estudiados mediante modelos estocáticos.

Como vemos existen una gran variedad de variables a tener en cuenta para formular un estudio matemático de una epidemia. En este articulo vamos a centrarnos en dos modelos sencilos:
\begin{enumerate}
\item Un modelo donde consideramos no hay ni nacimineto, ni muertos y la poblacion es contantes y no hay dependencia espacial. 
\end{enumerate}

\section{Modelo matematico de epidemias sin dependencia espacial}
En esta primera sección vamos a ver un modelo matematico deterministico muy sencillo. 
\section{Modelo de agentes}


\section{Virus de Marburg}
Fiebre hemorrágica de Marburgo

Nota descriptiva 
Noviembre de 2012

Datos y cifras

El virus de Marburgo causa en el ser humano una fiebre hemorrágica grave.
Las tasas de letalidad de los brotes de fiebre hemorrágica de Marburgo (FHM) han oscilado entre el 24% y el 88%.
Se considera que los huéspedes naturales del virus de Marburgo son los murciélagos de la fruta Rousettus aegypti, de la familia Pteropodidae. El virus de Marburgo se transmite de los murciélagos de la fruta a los seres humanos, y se propaga entre estos por transmisión de persona a persona.
No hay vacunas ni tratamientos antivíricos específicos.
El virus de Marburgo es el agente causal de la FHM, enfermedad cuya tasa de letalidad puede llegar al 88%. La FHM se identificó por vez primera en 1967 tras brotes simultáneos en Marburgo y Frankfurt (Alemania) y Belgrado (Serbia).

Los virus de Marburgo y del Ebola son los dos miembros de la familia Filoviridae (filovirus). Aunque son causadas por virus diferentes, las dos enfermedades (las fiebres hemorrágicas de Marburgo y del Ebola) son similares desde el punto de vista clínico. Ambas son raras, pero pueden ocasionar brotes dramáticos con elevadas tasas de letalidad.

Brotes

Dos grandes brotes que ocurrieron simultáneamente en Marburgo y Frankfurt (Alemania) y en Belgrado (Serbia) en 1967 llevaron a la identificación de la enfermedad por vez primera. El brote se asoció a trabajos de laboratorio con monos verdes africanos (Cercopithecus aethiops) importados de Uganda. Porteriormente se han notificado brotes y casos esporádicos en Angola, Kenya, la República Democrática del Congo, Sudáfrica (en una persona que había viajado recientemente a Zimbabwe) y Uganda. En 2008 se notificaron dos casos independientes en viajeros que había visitado en Uganda una cueva habitada por colonias de murciélagos Rousettus.

Transmisión

Originalmente, la infección humana se debe a la exposición prolongada a minas o cuevas habitadas por colonias de murciélagos Rousettus.

La transmisión se hace sobre todo de persona a persona por contacto estrecho con sangre, secreciones, órganos u otros líquidos corporales de personas infectadas. Las ceremonias funerarias en que los dolientes tienen contacto directo con el cuerpo del difunto pueden desempeñar un papel importante en la transmisión del virus de Marburgo. Puede haber transmisión por semen infectado hasta 7 semanas después de la recuperación clínica.

Se han descrito casos de transmisión al personal sanitario que atiende a los pacientes con FHM a través del contacto estrecho sin precauciones adecuadas de control de la infección. La transmisión por equipo de inyección contaminado o por pinchazos con agujas se asocia a una mayor gravedad de la enfermedad, deterioro rápido y, posiblemente, mayor tasa de letalidad.

Signos y síntomas

El periodo de incubación (intervalo entre la infección y la aparición de los síntomas) oscila entre 2 y 21 días.

La enfermedad causada por el virus de Marburgo empieza brucamente, con fiebre elevada, cefalea intensa y gran malestar. Los dolores musculares son frecuentes. Al tercer día pueden aparecer diarrea acuosa intensa, dolor y cólicos abdominales, náuseas y vómitos. La diarrea puede persistir una semana. En esta fase los pacientes tienen un aspecto que se ha descrito como “de fantasmas”, con hundimiento de los ojos, facies inexpresiva y letargo extremo. En el brote europeo de 1967 la mayoría de los pacientes presentaron una erupción cutánea no pruriginosa 2 a 7 días después del inicio de los síntomas.

Muchos pacientes tienen manifestaciones hemorrágicas graves a los 5 a 7 días, y los casos mortales suelen presentar alguna forma de hemorragia, a menudo en múltiples órganos. La presencia de sangre fresca en los vómitos y las heces suele acompañarse de sangrado por la nariz, encías y vagina. El sangrado espontáneo en los lugares de venopunción (para administración intravenosa de líquidos o extracción de muestras de sangre) puede ser especialmente problemático. Durante la fase grave de la enfermedad los pacientes tienen fiebre elevada persistente. La afectación del sistema nervioso central puede producir confusión, irritabilidad y agresividad. Ocasionalmente se han descrito casos de orquitis en la fase tardía de la enfermedad (15 días).

En los casos mortales el óbito suele producirse a los 8 a 9 días del inicio de los síntomas, generalmente precedido de grandes pérdidas de sangre y choque.

Diagnóstico

Entre los diagnósticos diferenciales se incluyen el paludismo, la fiebre tifoidea, la shigelosis, el cólera, la leptospirosis, la peste, la rickettsiosis, la fiebre recurrente, la meningitis, la hepatitis y otras fiebres hemorrágicas víricas.

El diagnóstico definitivo de la infección por el virus de Marburgo solo puede establecerse en el laboratorio, mediante diferentes pruebas:

inmunoadsorción enzimática (ELISA)
detección de antígenos
neutralización
reacción en cadena de la polimerasa con retrotranscriptasa (PCR-RT)
aislamiento del virus mediante cultivo celular.
Las pruebas con muestras clínicas suponen un enorme riesgo de contaminación y solo deben realizarse en condiciones de máxima contención biológica.

Treatamiento y vacunas

Los casos graves necesitan un tratamiento de sostén intensivo, pues suelen necesitar líquidos intravenosos o rehidratación oral con soluciones electrolíticas.

Todavía no hay tratamientos ni vacunas específicas para la FHM. Se están probando varias vacunas candidatas, pero pueden pasar varios años hasta que se disponga de una. En los estudios de laboratorio se han obtenido resultados prometedores con nuevos tratamientos farmacológicos que se encuentran en fase de investigación.

Huéspedes naturales del virus de Marburgo

Se considera que los huéspedes naturales del virus de Marburgo en África son los murciélagos de la fruta de la familia Pteropodidae, y en particular las especies pertenecientes al género Rousettus aegyptiacus . Los murciélagos no padecen enfermedad detectable. En consecuencia, la distribución geográfica del virus de Marburgo podría coincidir con la de los murciélagos Rousettus .

El virus de Marburgo en los animales

Los monos veredes africanos (Cercopithecus aethiops) importados de Uganda fueron la fuente de la infección humana en el primer brote de FHM.

La inoculación experimental al cerdo de diferentes virus del Ebola ha revelado que estos animales pueden infectarse por filovirus y que eliminan el virus. Por consiguiente, el cerdo puede considerarse como un posible huésped amplificador en los brotes de FHM. Aunque todavía no se ha confirmado una asociación entre otros animales domésticos y los brotes de filovirus, por precaución deben considerarse como potenciales huéspedes amplificadores hasta que se demuestre lo contrario.

Prevención

Medidas de precaución para las granjas de cerdos en las zonas endémicas
Son necesarias medidas de precaución en las granjas de cerdos de África para evitar que estos animales se infecten por contacto con los murciélagos de la fruta. Esas infecciones podrían amplificar el virus y causar brotes de FHM o contribuir a ellos.

Reducción del riesgo de infección humana
En ausencia de tratamientos y vacunas humanas eficaces, la sensibilización sobre los factores de riesgo de infección por el virus de Marburgo y las medidas de protección que se pueden adoptar para reducir la exposición humana al virus son las dos únicas formas de reducir las infecciones humanas y las muertes por esta causa.

En los brotes de FHM en África, los mensajes educativos de salud pública para reducir el riesgo deberían centrarse en:

La reducción del riesgo de transmisión del murciélago al ser humano a través de la exposición prolongada a minas o cuevas habitadas por colonias de murciélagos de la fruta. Durante el trabajo, las actividades de investigación o las visitas turísticas a esos lugares deberían utilizarse guantes y otras prendas de protección adecuadas, tales como mascarillas.
La reducción del riesgo de transmisión de persona a persona en la comunidad por contacto directo o íntimo con pacientes infectados, y en particular con sus líquidos corporales. Debe evitarse el contacto físico estrecho con pacientes infectados por el virus de Marburgo. Quienes cuiden de los pacientes en el domicilio deben utilizar guantes y otras prendas de protección personal adecuadas, además de lavarse las manos regularmente. El lavado de las manos se aplica también a las visitas de pacientes hospitalizados.
Las comunidades afectadas por el virus de Marburgo deben esforzarse por que la población esté bien informada sobre la naturaleza de la enfermedad y las medidas necesarias para contener los brotes, en especial la inhumación de los difuntos. Las víctimas de la FHM deben ser inhumadas rápidamente y en condiciones de seguridad.
Control de la infección en el entorno sanitario
La transmisión de persona a persona del virus de Marburgo se asocia principalmente al contacto directo con sangre y otros líquidos corporales, y se han descrito casos de transmisión asociados a la prestación de atención sanitaria cuando no se han observado las medidas apropiadas de control de la infección.

Los profesionales sanitarios que atiendan a pacientes con sospecha o confirmación de infección por virus de Marburgo deben adoptar medidas de control de la infección para evitar toda exposición a la sangre y otros líquidos corporales y el contacto directo sin protección con entornos posiblemente contaminados. Por consiguiente, la prestación de atención sanitaria a los casos sospechosos o confirmados de FHM requiere medidas de control específicas y un reforzamiento de las precauciones generales, en particular en lo que se refiere a la higiene de las manos, el uso de equipo de protección personal, las prácticas de inyección y de inhumación seguras.

Quienes trabajan en laboratorios también se encuentran en riesgo. Las muestras de casos humanos o animales sospechosos de FHM deben ser manipuladas por personal capacitado y procesadas en laboratorios equipados adecuadamente.

Respuesta de la OMS

La OMS se ha involucrado en todos los brotes de FHM habidos hasta ahora, proporcionado conocimientos técnicos y documentación para respaldar la investigación de la enfermedad y su control.

Las recomendaciones sobre el control de la infección durante la prestación de asistencia sanitaria a los pacientes con sospecha o confirmación de FHM figuran en la publicación: Interim infection control recommendations for care of patients with suspected or confirmed filovirus (Ebola, Marburg) Haemorrhagic Fever (marzo de 2008).

La OMS ha elaborado una lista de verificación de las precauciones generales en la atención sanitaria, cuyo objetivo consiste en reducir el riesgo de transmisión de patógenos transmitidos por la sangre y de otro tipo. Si su aplicación fuera universal, esas precauciones contribuirían a prevenir la mayoría de las transmisiones por exposición a sangre y otros líquidos corporales, y se recomienda su aplicación en la atención y tratamiento de todo tipo de pacientes, independientemente de su carácter infeccioso o no. Dichas precauciones incluyen las medidas básicas de control de las infecciones, tales como la higiene de las manos, el uso de equipo de protección personal para evitar el contacto directo con sangre y líquidos corporales, la prevención de las lesiones por pinchazo de agujas y otros instrumentos cortopunzantes, además de una serie de controles medioambientales.

Tabla: Cronología de los principales brotes de fiebre hemorrágica de Marburgo

Año	País	Subtipo de virus	Casos	Muertes	Tasa de letalidad
2008	Holanda	Marburg	1	1	100%
(ex-Uganda)
2008	USA	Marburg	1	0	0%
(ex-Uganda)
2007	Uganda	Marburg	4	2	50%
2005	Angola	Marburg	374	329	88%
1998-2000	República Democrática del Congo	Marburg	154	128	83%
1987	Kenya	Marburg	1	1	100%
1980	Kenya	Marburg	2	1	50%
1975	Sudáfrica	Marburg	3	1	33%
1967	Yugoslavia	Marburg	2	0	0%
1967	Alemania	Marburg	29	7	24%

%\bibliographystyle{plain}
%\bibliography{catenaria}



\newpage
%%% Local Variables: 
%%% mode: latex
%%% TeX-master: "matematicaseningenieria"
%%% End: 
