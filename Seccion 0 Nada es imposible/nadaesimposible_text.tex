
%{\normalfont\initfamily \fontsize{12mm}{12mm}\selectfont D}
\cappar Comecei a trabalhar a 24 de abril do 2007 para Soluciones de Gestión y Apoyo a Empresas s.l. a empresa em que desenvolvo as minhas funções laborais. Desde esse dia, até este presente ano 2014, tenho desempenhado diferentes cargos na mesma. Em cada um destes postos de trabalho, a minha grande prioridade tem sido sempre poder oferecer a Angola, uma vez que trabalhamos para ela, o melhor serviço possível e ao mesmo tempo sermos capazes de aplicar, tanto os meus conhecimentos de origem -- a engenharia industrial -- como outros tipos de conhecimentos e experiências adquiridos nos diferentes postos profissionais a nível nacional e internacional em anteriores empresas.
\begin{wrapfigure}{r}{0.40\textwidth} 
  \vspace{-21pt}
  \begin{figurebox}
   \vspace{20pt}
    \centering
    \includegraphics[height=0.35\textheight]{Enrique.jpg}\\
    Enrique Luis Ibáñez Conde\\ 
    {\small Diretor executivo}\\
    {\sl\small Soluciones de Gestión y Apoyo a Empresas s.l.}
    \vspace{1pt}
    %\vspace{0.1\textheight}
  \end{figurebox}
 \vspace{-20pt}
\end{wrapfigure}

Os fundadores da SAEMA começaram a prestar os seus serviços em Angola há mais de 23 anos. Isso significa que conviveram desde os tempos do conflito armado, passando pelo restabelecimento da paz e pelo processo das eleições democráticas até à atualidade. Nesses períodos, a SAEMA foi crescendo e fortalecendo-se com as suas prioridades focadas sempre no desenvolvimento de Angola.

Isso dá a oportunidade de ver, no dia de hoje, que a República de Angola é um país com vontade de crescer e, em e para este crescimento, a SAEMA tem disponibilizado todos os seus melhores recursos e desejos. Desde então, a SAEMA tem participado na construção e no início do funcionamento de serviços médicos como hospitais, centros de atenção, etc.; centrais de geração elétrica; linhas de transporte de Alta, Média e Baixa Tensão; sistemas de geração elétrica mediante equipamentos de captação fotovoltaica e instalações de captação, potabilização e distribuição de água para consumo humano. Tudo isso destinado, com a inestimável e imprescindível preocupação, promoção e financiamento do governo da República de Angola, a dar à população angolana um muito importante impulso ao seu desenvolvimento para conseguir uma alta melhoria na qualidade das suas vidas e do seu futuro.

O governo angolano tem tido a consideração de depositar a confiança na SAEMA para participar e cooperar nesta reestruturação e melhoria do país mediante os diferentes projetos e obras que anteriormente foram mencionadas nos campos sanitários, de energias e águas. Dentro do campo das energias, queremos salientar que energias renováveis como a fotovoltaica, deram a possibilidade de dispor de energia elétrica a núcleos povoados que não podem dispor dela, uma vez que ficam muito distantes das redes de distribuição elétrica do país.

É difícil pôr em números ou percentagens o quanto aumentou a qualidade de vida do povo angolano graças a toda a criação destes serviços e infraestruturas, mas sim podemos dizer que tem afetado a centenas de milhares deles e que nós, a SAEMA, estamos orgulhosos de ter podido contribuir nessa nossa parte.

E igual que o governo angolano continua a disponibilizar recursos, tanto económicos como  humanos, para fomentar e acelerar o crescimento do seu país, nós queremos cooperar nesses fins mediante as publicações periódicas desta revista. A citada revista, que para além de ser distribuída em formato papel, podem encontrá-la eletronicamente em www.revistasolucoes.com. Esperamos que com ela tenham acesso a temas científicos, bem como a tudo o relacionado com a engenharia, a informática e, ainda, possam passar um bocado agradável com os conteúdos que oferecemos.

Particularmente sinto-me satisfeito de que se tenha criado esta revista, e desejo que termine por participar nela, toda a comunidade angolana, e que sirva para que tenham uma outra via optativa destinada a incrementar a sua formação.


%\begin{wrapfigure}{r}{0.35\textwidth} 
%  \vspace{-25.5pt}
%  \begin{figurebox}
%    \vspace{20pt}
%    \centering
%    \includegraphics[height=0.28\textheight]{JoseManuelAlberto.jpg}\\
%    Enrique Luis Ibáñez Conde\\ 
%    Diretor executivo\\
%    {\sl SAEMA Projectos na área
%      eléctrica e médica, LDA}
%    %\vspace{0.1\textheight}
%  \end{figurebox}
%  \vspace{-10pt}
%
%\end{wrapfigure}


\newpage
%%% Local Variables: 
%%% mode: latex
%%% TeX-master: "nadaesimposible"
%%% End: 


