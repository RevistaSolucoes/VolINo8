\section*{Introducción}

En este artículo vamos a ver con mayor formalismo los tipos de datos
que podemos encontrar en Octave. Veremos como funcionan los comandos
\textit{do-until} y \textit{switch} y la notación de una función
\textit{function}. Finalmente hablaremos de como Octave resuelve
sistemas de ecuaciones no lineales.

\section{Tipos de datos, variables y expresiones}
 
\subsection{Tipos de datos}

En Octave podemos distinguir diferentes tipos de datos como son los
número reales, los números complejos, las matrices, las cadenas de
carácteres, los tipos estructurados y celdas. Más concretamente Octave
trabaja con 3 tipos tipos elementales de datos:
\begin{itemize}
\item Numéricos: enteros (con o sin signo, con 8,16,32 o 64 bits) on
  reales representados en coma flotanrte simple(32 bits) o doble(64
  bits).
\item Lógicos: que lo representan con 0 ó 1 mediante 8 bits.
\item Caracter:que se representa con 16 bits.
\end {itemize}

Veamos como sigue como con el comando \textit{typeinfo(x)} nos da la
información de que tipo de dato es $x$.
\begin{octavebox}
  \begin{verbatim}
a=1;
b=2+i;
c=[1 2 3;4 5 6];
d='obrigada';
e={a,b,c,d};
typeinfo(a)
typeinfo(b)
typeinfo(c)
typeinfo(d)
typeinfo(e)
\end{verbatim}
\end{octavebox}
La salida que se obtiene de estas instrucciones es:
\begin{octavebox}
  \begin{verbatim}
ans = scalar
ans = complex scalar
ans = matrix
ans = sq_string
ans = cell
  \end{verbatim}
\end{octavebox}
Si usamos la instrucción \textit{typeinfo()} nos devolverá la
información de todos los tipos de datos que cuenta Octave. Para mayir
información de tipos de datos y de construcción de datos estructurados
puedes remitirte al tutorial oficial que puedes encontrar aquí
\url{http://www.gnu.org/software/octave/octave.pdf}.  Para datos
numéricos tenemos un comando que a veces sçpuede ser de utilidad que
es el comando \textit{format} que sirve para cambiar el tipo de
visualización en pantalla de los datos. En general por defecto está
activado el formato \textit{short}, que muestra 5 dígitos
significativos. Mira, alguna de sus opciones como sigue,
\begin{octavebox}
  \begin{verbatim}
octave:12> pi
ans =     3.1416
octave:13> format long
octave:14> pi
ans =  3.14159265358979
octave:15> format long e
octave:16> pi
ans =    3.14159265358979e+00
octave:17> format long E
octave:18> pi
ans =    3.14159265358979E+00
octave:19> format rat
octave:20> pi
ans = 355/113
  \end{verbatim}
\end{octavebox}
 
\subsection{Variables}
En programación, una variable está formada por un espacio en el
sistema de almacenaje (memoria principal de un ordenador) y un nombre
simbólico (un identificador) que está asociado a dicho espacio. Ese
espacio contiene una cantidad o información conocida o desconocida, es
decir un valor. El nombre de la variable es la forma usual de
referirse al valor almacenado: esta separación entre nombre y
contenido permite que el nombre sea usado independientemente de la
información exacta que representa. El identificador, en el codigo
fuente de la computadora puede estar ligado a un valor durante el
tiempo de ejecución y el valor de la variable puede por lo tanto
cambiar durante el curso de la ejecución del programa. El concepto de
variables en computación puede no corresponder directamente al
concepto de variables en matemática. En Octave la notación es la
siguiente:
\begin{mybox}
variable=expresión
\end{mybox}
Cuando hemos comenzado una sesión de Octave el espacio de trabajo está
vacío. Algunos comandos que pueden ser de interés son los comandos:
\begin{itemize}
\item \textit{whos}: nos da la lista de las variables del espacio de trabajo con sus características: tamaños y tipos.
\item \textit{clear}: este comando borra el espacio de trabajo.
\item \textit{save} y \textit{load}: si queremos conservar ciertas
  variables para usar en otro momento, podemos grabarlas con el
  comando \text{save} y recuperarlas en cualquier monçmento con el
  comando \textit{load}. Estas se grabarán en el espacio de trabajo
  que estérs trabajando, Puedes ver donde es con el comando
  \textit{cd}
\end{itemize}
Veamos algunos ejemplos:
\begin{octavebox}
\begin{verbatim}
%  COMENZAMOS SESIÓN
whos
A=2
b=[pi;2;3]
whos
mivariable='mi nombre';
global mivariable
whos
load(mivariable)
\end{verbatim}
\end{octavebox}
En general las variables se mantendrán es el espacio de trabajo de la sesión. Cuando se inicia una sesión de trabajo en Octave es espacio de variables estará vacío e irñá almacenando las que se vayan generando. Una vez cerrado la sesión de trabajo las variables serán eliminadas. Si estás interesado en guardar alguna podrás hacerlo con el comando \textit{save} y recuperarla con el comando \textit{load}.

\subsection{Expresiones}
No vamos a entrar en detalle pero vamos a dar varios ejemplos de diferentes expresiones en Octave para que puedas hacerte a la idea del potencial y de sus distintos usos a la hora de programar. 
\begin{octavebox}
\begin{center}
\begin{tabular}{ccc}
Expresiones con índices &  Llamadas de funciones & Expresiones booleanas\\ \hline
$a=[1 \; 2;3 \; 4;5 \;6;7 \;8]$ & $sqrt(x^2+y^2)$ & $a \; \& \& \; b$ \\
$a([1 \; 3],2)$ & $ldivide(x,y)$ & $and(a,b)$ \\
$a(:,2)$ & $eye(3)$ & $not(a)$ \\
$a(1:2:end,:)$ & $rand()$ & $or(a,b)$ \\
 & $sqrt(x^2+y^2)$ & $a \& \& b++$ \\
\end{tabular}
\end{center}
\end{octavebox}

\section{Funciones}
Progrmas complicados en Octave pueden ser simplificados mediante funciones. Estas pueden ser definidas o bien en la línea de comandos de la sesión interactiva de Octave, o bien en un archivo con extensión .m. Recordad que los archivos de extensión .m pueden ser o bien scripts: que recogen diferentes ordenes que se ejecutan cuando hacemos el llamado en la loínea de comandos con el nombre del scripts o funciones, las cuales el nombre del archivo como el nombre de la función deben ser el mismo. La tipografía de una función puede  simplificarse como sigue en la Tabla \ref{tab:function}.
\begin{table}[!ht] 
\begin{mybox}%{c}{\textwidth} 
%\begin{table}  
\begin{tabular}{l|l|l}
    \begin{minipage}{0.2\linewidth}
\begin{verbatim} 
function name
body
endfunction
\end{verbatim}
\end{minipage}&
    \begin{minipage}{0.3\linewidth}
\begin{verbatim} 
function name (arg-list )
body
endfunction
\end{verbatim}    
    \end{minipage}&
    \begin{minipage}{0.4\linewidth}
\begin{verbatim} 
function [ret-list ] = name (arg-list )
body
endfunction
\end{verbatim}    
\end{minipage}\\
 \end{tabular}
%\end{table}
\end{mybox}\caption{De izquierda a derecha: sin input ni output, con input, con output e input}\label{tab:function}
\end{table}

Veamos tres ejemplos de funciones.

\begin{table}[!ht] 
\begin{mybox}%{c}{\textwidth} 
%\begin{table}  
\begin{tabular}{l|l}
    \begin{minipage}{0.4\linewidth}
\begin{verbatim} 
function simple1()
printf("Imaginação leva-nos
 a qualquer lado.");
endfunction
\end{verbatim}
\end{minipage}&
    \begin{minipage}{0.6\linewidth}
\begin{verbatim} 
function simple2(nombre,fecha)
  printf("Firmado \n %s \n %s\n",nombre,fecha);
endfunction;
\end{verbatim}    
    \end{minipage}\\
 \end{tabular}
%\end{table}
\end{mybox}\caption{De izquierda a derecha: sin input ni output, con input}\label{tab:function}
\end{table}

\begin{table}[!ht] 
\begin{mybox}%{c}{\textwidth} 
%\begin{table}  
\begin{tabular}{l}
    \begin{minipage}{1\linewidth}
\begin{verbatim} 
function [palabra longitud]=simple3(nombre)
  if (ischar(nombre)==1)
    palabra='True';
    longitud=length(nombre);
    else
      palabra="False";
      longitud=
   endif
endfunction
\end{verbatim}    
\end{minipage}\\
 \end{tabular}
%\end{table}
\end{mybox}\caption{Función con argumentos de entrda y de salida}\label{tab:function}
\end{table}

Mostrramos a continuación la llamada a funciones que hacemos y la salida: 
\begin{mybox}
\begin{tabular}{l}
    \begin{minipage}{0.7\linewidth}
\begin{verbatim} 
>>simple1()
Imaginação leva-nos a qualquer lado.

\end{verbatim}
\end{minipage}\\
 \begin{minipage}{\linewidth}
\begin{verbatim} 
>>simple2("maria","2 de septiembre de 1996")
Firmado 
 maria 
 2 de septiembre de 1996

\end{verbatim}
\end{minipage}\\
 \begin{minipage}{\linewidth}
\begin{verbatim} 
>>[sal1 sal2]=simple3("Antonio Yus")
sal1 = True
sal2 =  11
\end{verbatim}
\end{minipage}
 \end{tabular}
\end{mybox}

\section{Comandos: \textit{do-until} y \textit{switch}}

\subsection{switch}
Es muy comun realizar diferentes opciones dependiendo de los valores de una variable. Esto puede hacerse con el comando \textit{if}. Veamos el siguiente ejemplo:

\begin{octavebox}
\begin{verbatim}
if (X == 1)
   haz_algo ();
elseif (X == 2)
   haz_algo_distinto ();
else
   haz_algo_completamente_distinto ();
endif
\end{verbatim}
\end{octavebox}

Otra opcion para realizar este tipo de órdenes nes mediante el comando \textit{switch} como sigue:

\begin{octavebox}
\begin{verbatim}
switch (X)
 case 1
    haz_algo ();
 case 2
    haz_algo_distinto ();
 otherwise
    haz_algo_completamente_distinto ();
endswitch
\end{verbatim}
\end{octavebox}

\subsection{do-until}

El comando \textit{do-until} es muy parecido al comando \textit{while} que vimos en el número anterior pero con las siguientes diferencias: las repeticiones se realizan hasta que la condición sea cierta y el test de la condición se realiza al final no al principiio como el del \textit{while}.
\begin{octavebox}
\begin{verbatim}
do
body
until (condition )
\end{verbatim}
\end{octavebox}


\subsection{Solución al reto de la anterior semana}

La semana anterior planteamos el siguiente reto:

%\begin{wrapfigure}{r}{\textwidth} 
\begin{mybox}
   \centering {\fontsize{20}{10}\selectfont\color{red} El Reto}
  %\begin{tabular}{cc}
  %  \begin{minipage}{0.2\linewidth}
  %    {\fontsize{10}{5}\selectfont\color{red} El Reto}
  %  \end{minipage}&
  %  \begin{minipage}{0.8\linewidth}
  %     \begin{figurebox}

   \begin{center}\scalebox{0.64}{\input{gatito}}\end{center}
  % \end{figurebox}
Interpola mediante splines el lomo (parte roja) del gato.
  %  \end{minipage}
  %\end{tabular}\\
  Envía la solución a: \url{revista@revistasolucoes.com}
%\includegraphics[scale=0.4]{topimage.jpg}\\
    
\end{mybox}
%\end{wrapfigure}

¿Cómo os fue? Nosotros os damos la siguiente solución. Puedes encontrar el programa solución en la Sección de Códigos del formato y la salida que se tiene es la siguiente:
\begin{mybox}
   \centering {\fontsize{20}{10}\selectfont\color{red}Una solución al Reto}
  %\begin{tabular}{cc}
  %  \begin{minipage}{0.2\linewidth}
  %    {\fontsize{10}{5}\selectfont\color{red} El Reto}
  %  \end{minipage}&
  %  \begin{minipage}{0.8\linewidth}
  %     \begin{figurebox}

   \begin{center}\scalebox{0.64}{\input{gatitosplines}}\end{center}
  % \end{figurebox}
En azul, la aproximación del lomo mediante splines.
  %  \end{minipage}
  %\end{tabular}\\
  %\includegraphics[scale=0.4]{topimage.jpg}\\
    
\end{mybox}




%\vspace{3cm}
%\noindent
%\includegraphics[width=\textwidth]{pubmm2.png}

\newpage
%%% Local Variables: 
%%% mode: latex
%%% TeX-master: "informaticaeningenieria"
%%% End: 



