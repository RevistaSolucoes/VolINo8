\begin{tabular}{ll}
  \begin{minipage}{0.6\linewidth}
    {\bf\Large Em que é consiste o seu trabalho?}


\vspace{10pt}

O meu trabalho é ... faço carregamento de material de uma obra para outra, e com a
máquina faço terraplanagem do recinto onde se vai fazer o trabalho, faço valas, e outros
trabalhos por aí, que consistem em fazer com a máquina.

\vspace{20pt}

{\bf\Large Há quanto tempo que se dedica a isto?}

\vspace{10pt}

Já estou desde 2006, desde 2006 até hoje já são oito anos já nessa empresa, na empresa Saema, a fazer este tipo de trabalho.

\vspace{20pt}

{\bf\Large Do que é que mais gosta do seu trabalho?}

\vspace{10pt}

Do meu trabalho, basicamente, gosto de quase tudo, gosto de fazer as coisas bem, gosto de fazer as coisas que saem perfeitas, e de vez em quando tenho feito alguns papéis de encarregado.

\vspace{20pt}

{\bf\Large O que é que destacaria sobre a importância do seu trabalho?}

\vspace{10pt}

O mais importante do meu trabalho é quando termina uma obra. Quando terminamos uma obra, sim, é o mais importante do meu trabalho. 

\vspace{20pt}

{\bf\Large Relacionado com isto, que experiência tem quando termina uma obra, que experiência sente?}

\vspace{10pt}

Para mim, uma satisfação, para mim uma satisfação, alegria, é uma obra que terminou, assim a empresa vai subindo de patamar, uma obra que terminou, quando todo o mundo se sente satisfeito, não é? Mas para mim, é uma satisfação muito grande. 

\vspace{20pt}

{\bf\Large E quais são as reações das pessoas?}

\vspace{10pt}

Ficam satisfeitos, como veteranos que já estamos há mais anos na empresa, continuamos sempre na empresa, já vemos a empresa como o nosso filho, a nossa mãe, e já estamos há muitos anos. 
  \end{minipage}
  &
  \begin{minipage}{0.4\linewidth}
    \begin{figurebox}
    \vspace{20pt}
    \centering
    \includegraphics[height=0.3\textheight]{MFM.jpg}

     Miguel Francisco Miguel\\
    {\sl Operador de máquina e camionista}
    \vspace{20pt}
  \end{figurebox}
  \vspace{9cm}
  \end{minipage}
\end{tabular}
\newpage

%%% Local Variables: 
%%% mode: latex
%%% TeX-master: "solucionenaccion"
%%% End: 



